\begin{abstract}
    Introduction to technical debt....

    Architectural smells refer to violations of well known design principles that result in undesired dependencies, overblown size, and excessive coupling.
    Architectural smells have a negative influence on the maintainability levels of the system, making it harder to apply changes and add new functionality.
    Researchers have identified, described, and categorised several types of architectural smells over the past years.
    Subsequently, several research tools were born to automatically detect such smells starting from the source artefacts of a system.


    From a practical point of view, in order to properly be able to manage the technical debt generated by architectural smells, identification alone is not enough.
    To properly address the threat posed by architectural smells to the maintainbility of a system, practitioners require support for the prioritisation, quantification, and monitoring activities as well.
    Unfortunately, the literature on the topic is incomplete, and the tool support for these specific activities is missing.
    \textbf{This thesis reports reports on the investigation of how practitioners manage architectural smells in practice, what strategies exist to prioritise architectural smells refactoring, and design a methodology to quantify and monitor their contribution to technical debt.}

    There exist several types of architectural smells, therefore, to narrow down the scope of the investigation, we decide to focus on the four types that are the most prominent in the literature: Cyclic Dependency (CD), Hublike Dependency (HL), Unstable Dependency (UD), and God Component (GC).
    
    To address the problem stated above, we first opted to analyse the evolution of individual architectural smell instances in long-lived, open source systems by mining their software repositories.
    This allowed us to investigate prioritisation opportunities based on the history of an architectural smells.
    The results show how different smell types differ in multiple aspects, such as their growth rate, the importance of the affected elements over time in the dependency network of the system, and the time each instance affects the system.
    Based on these differences, we extrapolated that the refactoring of HL smells should be prioritised over CD smells and that many CD instances should have a low priority as they are likely to be organically removed in the upcoming releases.

    %Another finding of this study was that some of the smells are the result of intentional design decisions.
    These findings however are based on the mining software repositories, thus they can only tell half the story.
    Therefore, we decided to ask 21 software practitioners their opinion on architectural smells and what smells they think are more disruptive for the maintainability of a system.
    The findings show that practitioners think of certain smells as more impacful than others.
    One above all is the GC smell, that is perceived more detrimental the others, although the opinion on the HL smell was quite similar.
    These findings match the findings of the previous study.
   
    Another finding is that practitioners are aware of the smells, but they struggle to diligently implement good design practices with the resources (money and time) available to them.
    This motivated us to further investigate how architectural smells are introduced, and to broaden the scope, also how they co-occur, and evolve over time in an undustrial context.
    We set up an industrial case study to investigate these aspects in the context of large multinational company that works in C/C++, and verify whether the findings of the previous studies (that were all based on Java systems) apply to this context as well.
    The findings showed that CD smells, although they are not very persistent within the system, are precursors to other smell types.
    Therefore, CD instances can be used as a red flag that notifies practitioners that a component might arise a more severe architectural smell soon. Indeed, 94\% of HL smells, were also part of a cycle.
    
    The study also elicitated the opinion of software engineers and architects, who mentioned that the most problematic aspect of architectural smells are change ripple effects, that are unpredictable.
    This motivited us to investigate the relation between architectural smells and source code changes.
    After analysing over 30 open source Java systems, we ran a series of statistical tests to determine whether source code changes were more frequent, and larger in size, in components that were affected by smells than in non-smelly components.
    The findings show that this is indeed the case, and that all four architectural smells correlate with higher frequency of change of the affected part. 
    Interestingly, after talking to practitioners in the previous study we expected UD to be the smell with the highest correlation with changes. THis turned out, however, not to be the case as the HL smell beat UD by over 10\%.
    This means prioritising the refactoring of HL instances over the other smell types is a more advantageous approach for practitioners.

    The answers collected up until this point provide valuable insights on how architectural smells are preceived by practitioners, how they get introduced, how they co-occur, and how they correlate with the changes done in the system.
    These findings allowed us to extrapolate valuable information on what are the best strategies to prioritise smell instances and what makes one instance more severe than the other.
    Therefore, using the acquired knowledge, we created a machine learning model that ranks architectural smells based on their severity that performs rather well (97\% of instances are ranked above a less severe smell).
    This model makes it easier for practitioners to prioritise architectural smells, especially in very large systems where hundreds of instances can be detected.
    Additionally, using this model, we designed an approach to quantify the amount of technical debt generated by the architectural smells detected in a system.
    The validation of the model showed that in 71\% of the cases, practitioners agreed that the estimations provided were representative of the effort necessary to refactor the smell.

    With this study, we provided a first attempt at a comprehesinve solution to the repayment of the technical debt generated by architectural smells. 
    However, the findings of the study also hint at the fact that TD prioritisation (and consequently its repayment) is influenced by trade-offs with other quality attributes (e.g. availability).
    To further investigate this aspect, we set up a case study and talked to 21 practitioners.
    The findings show that practitioners prioritise run-time qualities over design-time qualities, and often do so implicitly, due to lack of monitoring tools. 
    


\end{abstract}

%Enabling technical debt repayment for architectural smells