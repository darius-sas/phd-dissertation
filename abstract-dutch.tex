\samenvatting{
Technical Debt (Technische schuld-TD) is een term die een verzameling ontwerp- en implementatie constructies beschrijft die ongeschikt zijn voor de lange termijn, maar aan de behoeftes van stakeholders voldoen op de korte termijn. De aanwezigheid van TD maakt het moeilijker voor software developers en architecten om software systemen te begrijpen, aan te passen en te onderhouden. Uiteindelijk kan een te grote hoeveelheid TD er zelfs toe leiden dan een project mislukt door te hoge onderhoudskosten.

Door de jaren heen hebben onderzoekers de definitie van TD verfijnd en hebben ze verschillende verschijningsvormen van TD geïdentificeerd. Het belangrijkste onderscheid tussen die vormen zit in het type artefact en het abstractieniveau dat bepaalde TD beïnvloedt. Architectural Technical Debt (ATD) is het type TD dat betrekking heeft op de architectuur van een systeem.

Architectural smells zijn een opvallend schadelijke vorm van ATD. Het zijn overtredingen van bekende ontwerpprincipes die resulteren in ongewenste afhankelijkheden, overmatige grootte, en overmatige coupling. Architectural smells hebben een negatieve invloed op de onderhoudbaarheid en de ontwikkelmogelijkheden van een systeem wat het ingewikkelder maakt om veranderingen door te voeren en nieuwe functionaliteit toe te voegen. Onderzoekers hebben over de jaren heen meerdere vormen van architectural smells geïdentificeerd, beschreven en gecategoriseerd. Hierop volgend zijn er onderzoeksapplicaties gemaakt die deze smells automatisch op kunnen sporen in de bronartefacten van een systeem.

Vanuit een praktisch oogpunt is identificatie niet genoeg om technical debt die door architectural smells gegenereerd wordt in bedwang te houden. Om het risico dat architectural smells vormen voor de onderhoudbaarheid van een systeem goed aan te kaarten, hebben ontwikkelaars en architecten ook ondersteuning nodig bij het prioriteren, quantificeren, aflossen, en monitoren van trends van architectural smells. Helaas is de literatuur over dit onderwerp incompleet en schieten de bestaande hulpapplicaties voor deze specifieke activiteiten tekort.

De nadruk van dit proefschrift ligt op het verbeteren van de ondersteuning voor professionals die architectural smells beheersbaar moeten houden. In het bijzonder focust het op ondersteuning verschaffen voor professionals die de hoeveelheid TD veroorzaakt door iedere architectural smell moeten quantificeren, de trend van TD over de tijd heen moeten monitoren, en beschikbare prioriteringsstrategiën moeten identificeren die het afbetalen van opgelopen TD makkelijker maken.

Er bestaan meerdere soorten architectural smells. Om de reikwijdte van het werk in te perken, besluiten we ons te richten op de vier prominentste typen uit de literatuur: Cyclic Dependency (CD), Hublike Dependency (HL), Unstable Dependency (UD) en God Component (GC).

Om de ondersteuning van architectural smells management te verbeteren kozen we allereerst om de evolutie van individuele architectural smells instanties te analyseren in langlevende open-source systemen door hun software repository te delven. Dit liet ons prioriteringsmogelijkheden onderzoeken op basis van de levensloop van architectural smells. De resultaten laten zien hoe verscheidene soorten smells op meerdere vlakken verschillen, waaronder hun groeisnelheid, het belang van de geraakte elementen over de tijd heen in het afhankelijkheidsnetwerk van het systeem, en de tijd dat iedere smell het systeem beïnvloedt. Op basis van deze verschillen extrapoleren we dat het refactoren van HL smells geprioriteerd moeten worden boven CD smells, en dat vele instanties van CD smells een lage prioriteit zouden moeten krijgen, aangezien ze hoogstwaarschijnlijk op natuurlijke wijze verwijderd worden in toekomstige releases.

Deze bevindingen zijn echter gebaseerd op het delven van software repositories, waardoor ze slechts de helft van het verhaal kunnen vertellen. Om de andere helft te onderzoeken, besloten we om 21 software professionals te vragen naar hun mening over architectural smells en van welke smells zij denken dat die de onderhoudbaarheid van systemen het meest saboteren. De bevindingen laten zien dat professionals denken dat bepaalde smells meer impact hebben dan andere. Voornamelijk GC smells worden gezien als invloedrijk, alhoewel HL smells vergelijkbaar werden beoordeeld.

Door het praten met professionals kwamen we er ook achter dat ze moeite hebben met het in de gaten houden van architectural smells, hun oorsprong en eventuele onderlinge overlap. Dit heeft ons ertoe aangezet om ook te onderzoeken waar architectural smells vandaan komen, hoe vaak ze samen optreden, en hoe ze evolueren in de context van het bedrijfsleven. We hebben een case study opgezet om deze aspecten te onderzoeken, in de context van een groot internationaal bedrijf dat werkt in C/C++, en om te bevestigen dat onze eerdere bevindingen (die gebaseerd waren op Java systemen) ook hier van toepassing zijn. De resultaten laten zien dat CD smells, ook al zijn ze redelijk vergankelijk binnen het systeem, voorspellers zijn van andere soorten smells. CD smell instanties kunnen dus gezien worden als een rode vlag die professionals waarschuwt dat een component binnenkort ergere architectural smells zou kunnen gaan bevatten. Daarnaast bleken 94\% van HL smells onderdeel te zijn van een cyclus. 

Verder hebben we, als onderdeel van hetzelfde onderzoek, software engineers en architecten bevraagd. Zij kaartten aan dat kettingsreacties van veranderingen, die onvoorspelbaar zijn, het meest problematische aspect van architectural smells zijn. Daarom hebben we ook onderzocht wat de relatie tussen architectural smells en code-aanpassingen is. Na het analyseren van 30 open-source Java systemen hebben we meerdere statistische tests uitgevoerd om te bepalen of code-aanpassingen vaker voorkwamen, en groter waren, in componenten die door een architectural smell geraakt werden dan in componenten die niet geraakt werden door architectural smells. Dit werd door de resultaten bevestigd. Alle vier de soorten architectural smells correleren met meer aanpassingen in een erdoor geraakt component. Interessant genoeg was dat, na het praten met software engineers en architecten, we hadden verwacht dat UD smells de hoogste correlatie hadden met veranderfrequentie. Dit bleek echter niet het geval. HL smells hadden namelijk een 10\% hogere veranderfrequentie dan UD smells. Dit betekent dat het voordeliger is voor professionals om voorrang te geven aan het refactoren van HL smells in vergelijking met andere soorten smells. 

De antwoorden die we tot nu toe verzameld hebben bieden waardevolle inzichten in hoe architectural smells gezien worden door professionals, hoe ze onstaan, wanneer ze samen voorkomen, en hoe ze correleren met veranderingen van het systeem. Dankzij deze resultaten hebben we waardevolle informatie kunnen extrapoleren over de beste prioriteringsstrategieën en over wat een wat bepaalde architectural smell instanties erger maakt dan andere. Gebaseerd op de verkregen kennis hebben we een machine learning model gemaakt dat architectural smells sorteert gebaseerd op ergheid. Dit model functioneert vrij goed: 97\% van de smells worden geordend boven een minder erge smell). Het stelt professionals in staat om architectural smells te ordenen, vooral in gigantische systemen waar honderden smell instanties te vinden zijn. Daarnaast hebben we, door gebruik te maken van het machine learning model, een aanpak opgesteld om de hoeveelheid TD in te schatten, die veroorzaakt is door gedetecteerde architectural smells in een systeem. De validatie van het model wijst uit dat in 71\% van de gevallen, professionals zich konden vinden in de inschattingen , die waren afgegeven door het model over hoeveel werk het refactoren van een smell zou zijn.

Tot slot hebben we geleerd dat professionals TD prioritering (en daaruit volgend de aflossing ervan) niet alleen door TD instanties wordt beinvloed, maar ook door andere kwaliteitsattributen (zoals beschikbaarheid). Dit dwingt professionals constant om onderhoudbaarheid op te offeren. We hebben dit aspect verder onderzocht en gepraat met 21 professionals. We hebben geleerd dat professionals run-time kwaliteiten de voorkeur verlenen boven design-time kwaliteiten en dat dit vaak onbewust gebeurd door het gebrek aan monitoring tools. 
}