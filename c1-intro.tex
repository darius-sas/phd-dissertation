\chapter{Introduction}\label{chap:introduction}

\epigraph{\emph{A program that is used and that, as an implementation of its specification, reflects some other reality, undergoes continuos change or becomes progressively less useful.
The change or decay process continues until it is judged more cost effective to replace the program with a recreated version.}}{--- Meir Lehman}

%\section{The relentless endeavor of software craftmanship}
The opening quote of this chapter is the first of the five laws of software evolution formulated by Lehman in the late 1970's \cite{Lehman1979}.
The law refers to the fact that all software is designed to operate in a specific enviroment and to satisfy a specific set of requirements. 
However, every enviroment, and every requirement, is bound to change eventually, rendering the software obsolete. %, unless it changes accordingly.
Therefore, the need of constantly adapting a software to new requirements, and shifts in the environment, is a \emph{relentless endeavor} that, over time, destabilises the sustainability of a software project.

A software project is sustainable if the project owner is capable of applying whatever valuable change they ought to make, timely \cite{Winters2020}.
However, decisions and implementation choices made early on in the project's lifetime inevitably affect the decisions we have to make in the present, making them harder.
Over time, as the system grows old, our capability of adapting the software to new requirements and changes in the enviroment grows narrower, and making changes becomes more expensive.
Eventually, the system becomes unsustainable and the second part of Lehman's law comes into play.

In 1992, Cunningham first introduced the concept of \emph{technical debt} \cite{Cunningham1992}. 
Technical debt (TD) is a powerful metaphor that allows us to better think and communicate about sustainability and Lehman's law.
The term was first coined to indicate the necessity of releasing software that while it works perfectly, it does not meet the criteria of long-term sustainable software. 
Cunningham himself calls this an ``unmasterable program'' that is ``dangerous'' unless the debt is repaid.
Since the metaphor is easily digestable for people with non-technical backgrounds, it is a powerful tool

% TD is the cause of unsustainable software projects.

A modern definition of TD is the following: TD reflects the technical compromises that software practitioners make in order to achieve a short-term advantage at the expense of creating a technical context that increases complexity and cost in the long-term \cite{Avgeriou2016}. 

Over time, several types of debt were identified... 
AD are one of the main source of TD, which renders ATD one of the most infamous types of TD...
AS are defined as AD that negatively impact the quality... (garcia definition)
AS also 
AS seem to be an interesting way of approaching AD 


\section{Technical Debt}
The concept of \textit{debt} in Software Engineering was initially introduced by Cunningham in 1992 \cite{Cunningham1992}.
Technical debt (TD) reflects the technical compromises that software practitioners make in order to achieve a short-term advantage at the expense of creating a technical context that increases complexity and cost in the long-term \cite{Avgeriou2016}.  

Hence, a company can get into debt and, as long as it is aware of the debt and is planning to repay it in the medium-term period, leverage on it to temporarily increase productivity.
However, if the company is not aware that it is accruing TD, or does not repay it on time, the amount of interest may become too high, causing the failure of the project due to the huge cost of implementing changes.

During the course of the years, the metaphor has been extended by the research community and has assumed a wider meaning, englobing several aspects of the software development process like architecture, design, requirements, testing and documentation \cite{brown_managing_2010}.
It is very hard to state a single definition that incorporates all the facets of such a big metaphor. 
But the current literature has explored the concept in all of its vastness and has proposed and analyzed multiple taxonomies and types of TD.
A common way of categorising TD is by the type of artefacts it affects. Using this approach, Li et al. identified several different types of TD \cite{li_systematic_2015}, namely \emph{Requirements TD}, \emph{Architectural TD}, \emph{Design TD}, \emph{Code TD}, \emph{Test TD}, \emph{Build TD}, \emph{Documentation TD}, \emph{Infrastructure TD}, and \emph{Versioning TD}.



\section{Architectural smells}

\section{The project SDK4ED}

\section{Research design}
\subsection{Problem statement}
\subsection{Design science and research methodology}
\subsection{Overview of this dissertation}

