\chapter{Introduction}\label{chap:introduction}

\epigraph{\emph{A program that is used and that, as an implementation of its specification, reflects some other reality, undergoes continuos change or becomes progressively less useful.
The change or decay process continues until it is judged more cost effective to replace the program with a recreated version.}}{--- Meir Lehman}

%\section{The relentless endeavor of software craftmanship}
\lettrine{T}{he} opening quote of this chapter is the first of the five laws of software evolution formulated by Lehman in the late 1970's \cite{Lehman1979}.
The law refers to the fact that all software is designed to operate in a specific enviroment and to satisfy a specific set of requirements. 
However, every enviroment, and every requirement, is bound to change eventually, rendering the software obsolete. %, unless it changes accordingly.
Therefore, the need of constantly adapting a software to new requirements, and shifts in the environment, is a \emph{relentless endeavor} that, over time, destabilises the sustainability of a software project.

A software project is sustainable if the project owner is capable of applying whatever valuable change they ought to make, timely \cite{Winters2020}.
However, decisions and implementation choices made early on in the project's lifetime inevitably affect the decisions we have to make in the present, making them harder.
Over time, as the system grows old, our capability of adapting the software to new requirements and changes in the enviroment grows narrower, and making changes becomes more expensive.
Eventually, the system becomes unsustainable and the second part of Lehman's first law of software evolution comes into play.

In 1992, Cunningham first introduced the concept of \emph{technical debt} (TD) \cite{Cunningham1992}. 
The term was first coined to indicate the necessity of releasing software that, while it works perfectly, it does not meet the criteria of long-term sustainable software. 
Cunningham himself calls this an \emph{``unmasterable program''} that is \emph{``dangerous''} unless the \emph{debt} is repaid.
Unfortunately, TD repayment is not always feasible, as software practitioners have to work with limited time and budget, resulting in most of TD not being repaid \cite{Digkas2018}.
The time spent on not-quite-right code counts as \emph{interest} on that debt \cite{Cunningham1992}, making software projects more expensive to maintain.

Technical debt is a powerful metaphor that, essentially, conveys the importance of sustainable software -- and Lehman's first law of software evolution -- in terms that are easy to understand and communicate to others. 

While technical debt is the \emph{leitmotif} of this dissertation, it is not the main topic.
The main topic are \emph{architectural smells} (AS).  I will describe AS later on, but for now, it's sufficient to know that AS are a 


Over time, researchers have identified several types of TD, based on the 


This dissertation deals with AS: how they are introduced, how they evolve, and how they impact software maintenance.
TD instead, is the \emph{leitmotif} of this dissertation.



Over time, several types of debt were identified... 
AD are one of the main source of TD, which renders ATD one of the most infamous types of TD...
AS are defined as AD that negatively impact the quality... (garcia definition)
AS also 
AS seem to be an interesting way of approaching AD 



\section{Technical Debt}
A modern definition of TD is the following: TD reflects the technical compromises that software practitioners make in order to achieve a short-term advantage at the expense of creating a technical context that increases complexity and cost in the long-term \cite{Avgeriou2016}. 

The concept of \textit{debt} in Software Engineering was initially introduced by Cunningham in 1992 \cite{Cunningham1992}.
Technical debt (TD) reflects the technical compromises that software practitioners make in order to achieve a short-term advantage at the expense of creating a technical context that increases complexity and cost in the long-term \cite{Avgeriou2016}.  

Hence, a company can get into debt and, as long as it is aware of the debt and is planning to repay it in the medium-term period, leverage on it to temporarily increase productivity.
However, if the company is not aware that it is accruing TD, or does not repay it on time, the amount of interest may become too high, causing the failure of the project due to the huge cost of implementing changes.

During the course of the years, the metaphor has been extended by the research community and has assumed a wider meaning, englobing several aspects of the software development process like architecture, design, requirements, testing and documentation \cite{brown_managing_2010}.
It is very hard to state a single definition that incorporates all the facets of such a big metaphor. 
But the current literature has explored the concept in all of its vastness and has proposed and analyzed multiple taxonomies and types of TD.
A common way of categorising TD is by the type of artefacts it affects. Using this approach, Li et al. identified several different types of TD \cite{li_systematic_2015}, namely \emph{Requirements TD}, \emph{Architectural TD}, \emph{Design TD}, \emph{Code TD}, \emph{Test TD}, \emph{Build TD}, \emph{Documentation TD}, \emph{Infrastructure TD}, and \emph{Versioning TD}.



\section{Architectural smells}

% \subsection{Why smells over existing metric?}

\section{The project SDK4ED}

\section{Research design}
\subsection{Problem statement}
In the early years of the research on architectural smells, researchers focused on identifying new types of smells, propose a  theoretical definition, define a detection rule, and finally describe their impact on software maintenance from a theoretical point of view \cite{Lippert2006,Garcia2009,Mo2015,Le2016,Arcelli2016}.
From there, researchers moved on to study how smells impact software maintenance in open source systems \cite{Choudhary2016,Xiao2016,Le2018}, study their interaction with other types of smells \cite{Sharma2017,Arcelli2019}, and even predict their introduction in future releases \cite{Arcelli2019b}.
All of these studies offer a great insight into the theoretical nature of architectural smells, and some of them even provide limited evidence to support whether the hypothesized negative effects on software maintenance are detectable in reality.
However, there are still three important research threads that are either incomplete or not investigated at all:
\begin{enumerate}
    \item there are no specific details on how individual architectural smell instances evolve over time. This means that it is not clear how architectural smells are \emph{introduced}, how long they \emph{persist} within the system, and whether any TD interest is paid on the components affected by architectural smells; 
    \item no studies investigate how architectural smells are perceived by practitioners. This means that thare is little to no empirical evidence on how architectural smells concretely affect the work of software developer and architects;
    \item only a few studies investigate the possibility to estimate the amount of technical debt principal generated by architectural smells. If architectural smells are to be used by practitioners to manage TD, this is an instrumental step to achieve that. The approaches described by existing literature are not properly validated with industrial practitioners.
\end{enumerate}


Architectural decisions are the most significant source of TD
Making decisions without being properly informed makes it easier to introduce TD.
AS are decisions [Garcia]
Over the past decade, literature has several works studying AS. Make list of studies and what they achieved, briefly.
Several questions remain unanswered however, such as "How are AS perceived?" "How do they evolve?", "How do they correlate with changes in the system?".
These questions are of paramount importance in order to be able to provide practitioners a data-driven approach to manage AS.
The following statement summarises the 

\subsection{Design science and research methodology}
\subsection{Overview of this dissertation}

