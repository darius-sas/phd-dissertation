\documentclass[b5paper]{article}
\usepackage[margin=0.9in,top=0in]{geometry}
\usepackage{palatino}
\usepackage{enumerate}
\usepackage{xcolor}

\author{\Large{Darius Sas}} % TODO Update Title
\title{Stellingen\\{\large{behorende bij het proefschrift}}\\[.2cm]
Managing Technical Debt: Prioritising and Quantifying Achitectural Smells \\[.2cm]{\large{van}}}
\date{}

\begin{document}
\maketitle
\thispagestyle{empty}

\begin{enumerate}
\setlength{\itemsep}{3mm}
\begin{large}

\item The architectural smells detected in open source systems evolve rapidly: 50\% of Cyclic Dependencies are replaced by a new instance within the next 3 releases. {\color{teal}{Chapter 2}}

\item The number of artefacts affected by a Cyclic Dependency tends to stay constant over time. The opposite holds for Hublike Dependency smell, which grows in 65\% of the cases. {\color{teal}{Chapter 2}}

\item Software practitioners rank change propagation as the most disruptive effect on maintainability associated to architectural smells. Difficulty to understand and properly orient in the code follow next. {\color{teal}{Chapter 3}}

\item The God Component and Hublike Dependency smells were considered the most severe smells. {\color{teal}{Chapter 3}}

\item As opposed to open source systems, architectural smells in industrial systems of all types exhibit longer lifespans. {\color{teal}{Chapter 4}}

\item In industrial software systems, Cyclic dependencies are precursors to other, more severe architectural smells. Conversely, Hublike dependencies are the most common successor. {\color{teal}{Chapter 4}}

\item Source code artefacts affected by architectural smells are subject to  more frequent and larger changes than non-affected artefacts.  {\color{teal}{Chapter 5}}

\item The introduction of an architectural smell follows with an increase in the change frequency in the affected artefact. {\color{teal}{Chapter 5}}
% Different phrasing
% \item Architectural smells' introduction leads to an increase in the change frequency of the affected artefacts. {\color{teal}{Chapter 5}}

\item Under certain conditions, applying large refactorings in long-lived systems is almost impossible for practitioners, as they would risk to compromise vital run-time quality attributes. 
This limits the use of all TD management tools.  {\color{teal}{Chapter 6}}

\item Trade-offs between design-time and run-time quality attributes
are very common, but they are often implicit, due to the lack of adequate monitoring tools and practices. {\color{teal}{Chapter 7}}


\end{large}

\end{enumerate}

\end{document}
